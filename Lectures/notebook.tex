
% Default to the notebook output style

    


% Inherit from the specified cell style.




    
\documentclass[11pt]{article}

    
    
    \usepackage[T1]{fontenc}
    % Nicer default font (+ math font) than Computer Modern for most use cases
    \usepackage{mathpazo}

    % Basic figure setup, for now with no caption control since it's done
    % automatically by Pandoc (which extracts ![](path) syntax from Markdown).
    \usepackage{graphicx}
    % We will generate all images so they have a width \maxwidth. This means
    % that they will get their normal width if they fit onto the page, but
    % are scaled down if they would overflow the margins.
    \makeatletter
    \def\maxwidth{\ifdim\Gin@nat@width>\linewidth\linewidth
    \else\Gin@nat@width\fi}
    \makeatother
    \let\Oldincludegraphics\includegraphics
    % Set max figure width to be 80% of text width, for now hardcoded.
    \renewcommand{\includegraphics}[1]{\Oldincludegraphics[width=.8\maxwidth]{#1}}
    % Ensure that by default, figures have no caption (until we provide a
    % proper Figure object with a Caption API and a way to capture that
    % in the conversion process - todo).
    \usepackage{caption}
    \DeclareCaptionLabelFormat{nolabel}{}
    \captionsetup{labelformat=nolabel}

    \usepackage{adjustbox} % Used to constrain images to a maximum size 
    \usepackage{xcolor} % Allow colors to be defined
    \usepackage{enumerate} % Needed for markdown enumerations to work
    \usepackage{geometry} % Used to adjust the document margins
    \usepackage{amsmath} % Equations
    \usepackage{amssymb} % Equations
    \usepackage{textcomp} % defines textquotesingle
    % Hack from http://tex.stackexchange.com/a/47451/13684:
    \AtBeginDocument{%
        \def\PYZsq{\textquotesingle}% Upright quotes in Pygmentized code
    }
    \usepackage{upquote} % Upright quotes for verbatim code
    \usepackage{eurosym} % defines \euro
    \usepackage[mathletters]{ucs} % Extended unicode (utf-8) support
    \usepackage[utf8x]{inputenc} % Allow utf-8 characters in the tex document
    \usepackage{fancyvrb} % verbatim replacement that allows latex
    \usepackage{grffile} % extends the file name processing of package graphics 
                         % to support a larger range 
    % The hyperref package gives us a pdf with properly built
    % internal navigation ('pdf bookmarks' for the table of contents,
    % internal cross-reference links, web links for URLs, etc.)
    \usepackage{hyperref}
    \usepackage{longtable} % longtable support required by pandoc >1.10
    \usepackage{booktabs}  % table support for pandoc > 1.12.2
    \usepackage[inline]{enumitem} % IRkernel/repr support (it uses the enumerate* environment)
    \usepackage[normalem]{ulem} % ulem is needed to support strikethroughs (\sout)
                                % normalem makes italics be italics, not underlines
    

    
    
    % Colors for the hyperref package
    \definecolor{urlcolor}{rgb}{0,.145,.698}
    \definecolor{linkcolor}{rgb}{.71,0.21,0.01}
    \definecolor{citecolor}{rgb}{.12,.54,.11}

    % ANSI colors
    \definecolor{ansi-black}{HTML}{3E424D}
    \definecolor{ansi-black-intense}{HTML}{282C36}
    \definecolor{ansi-red}{HTML}{E75C58}
    \definecolor{ansi-red-intense}{HTML}{B22B31}
    \definecolor{ansi-green}{HTML}{00A250}
    \definecolor{ansi-green-intense}{HTML}{007427}
    \definecolor{ansi-yellow}{HTML}{DDB62B}
    \definecolor{ansi-yellow-intense}{HTML}{B27D12}
    \definecolor{ansi-blue}{HTML}{208FFB}
    \definecolor{ansi-blue-intense}{HTML}{0065CA}
    \definecolor{ansi-magenta}{HTML}{D160C4}
    \definecolor{ansi-magenta-intense}{HTML}{A03196}
    \definecolor{ansi-cyan}{HTML}{60C6C8}
    \definecolor{ansi-cyan-intense}{HTML}{258F8F}
    \definecolor{ansi-white}{HTML}{C5C1B4}
    \definecolor{ansi-white-intense}{HTML}{A1A6B2}

    % commands and environments needed by pandoc snippets
    % extracted from the output of `pandoc -s`
    \providecommand{\tightlist}{%
      \setlength{\itemsep}{0pt}\setlength{\parskip}{0pt}}
    \DefineVerbatimEnvironment{Highlighting}{Verbatim}{commandchars=\\\{\}}
    % Add ',fontsize=\small' for more characters per line
    \newenvironment{Shaded}{}{}
    \newcommand{\KeywordTok}[1]{\textcolor[rgb]{0.00,0.44,0.13}{\textbf{{#1}}}}
    \newcommand{\DataTypeTok}[1]{\textcolor[rgb]{0.56,0.13,0.00}{{#1}}}
    \newcommand{\DecValTok}[1]{\textcolor[rgb]{0.25,0.63,0.44}{{#1}}}
    \newcommand{\BaseNTok}[1]{\textcolor[rgb]{0.25,0.63,0.44}{{#1}}}
    \newcommand{\FloatTok}[1]{\textcolor[rgb]{0.25,0.63,0.44}{{#1}}}
    \newcommand{\CharTok}[1]{\textcolor[rgb]{0.25,0.44,0.63}{{#1}}}
    \newcommand{\StringTok}[1]{\textcolor[rgb]{0.25,0.44,0.63}{{#1}}}
    \newcommand{\CommentTok}[1]{\textcolor[rgb]{0.38,0.63,0.69}{\textit{{#1}}}}
    \newcommand{\OtherTok}[1]{\textcolor[rgb]{0.00,0.44,0.13}{{#1}}}
    \newcommand{\AlertTok}[1]{\textcolor[rgb]{1.00,0.00,0.00}{\textbf{{#1}}}}
    \newcommand{\FunctionTok}[1]{\textcolor[rgb]{0.02,0.16,0.49}{{#1}}}
    \newcommand{\RegionMarkerTok}[1]{{#1}}
    \newcommand{\ErrorTok}[1]{\textcolor[rgb]{1.00,0.00,0.00}{\textbf{{#1}}}}
    \newcommand{\NormalTok}[1]{{#1}}
    
    % Additional commands for more recent versions of Pandoc
    \newcommand{\ConstantTok}[1]{\textcolor[rgb]{0.53,0.00,0.00}{{#1}}}
    \newcommand{\SpecialCharTok}[1]{\textcolor[rgb]{0.25,0.44,0.63}{{#1}}}
    \newcommand{\VerbatimStringTok}[1]{\textcolor[rgb]{0.25,0.44,0.63}{{#1}}}
    \newcommand{\SpecialStringTok}[1]{\textcolor[rgb]{0.73,0.40,0.53}{{#1}}}
    \newcommand{\ImportTok}[1]{{#1}}
    \newcommand{\DocumentationTok}[1]{\textcolor[rgb]{0.73,0.13,0.13}{\textit{{#1}}}}
    \newcommand{\AnnotationTok}[1]{\textcolor[rgb]{0.38,0.63,0.69}{\textbf{\textit{{#1}}}}}
    \newcommand{\CommentVarTok}[1]{\textcolor[rgb]{0.38,0.63,0.69}{\textbf{\textit{{#1}}}}}
    \newcommand{\VariableTok}[1]{\textcolor[rgb]{0.10,0.09,0.49}{{#1}}}
    \newcommand{\ControlFlowTok}[1]{\textcolor[rgb]{0.00,0.44,0.13}{\textbf{{#1}}}}
    \newcommand{\OperatorTok}[1]{\textcolor[rgb]{0.40,0.40,0.40}{{#1}}}
    \newcommand{\BuiltInTok}[1]{{#1}}
    \newcommand{\ExtensionTok}[1]{{#1}}
    \newcommand{\PreprocessorTok}[1]{\textcolor[rgb]{0.74,0.48,0.00}{{#1}}}
    \newcommand{\AttributeTok}[1]{\textcolor[rgb]{0.49,0.56,0.16}{{#1}}}
    \newcommand{\InformationTok}[1]{\textcolor[rgb]{0.38,0.63,0.69}{\textbf{\textit{{#1}}}}}
    \newcommand{\WarningTok}[1]{\textcolor[rgb]{0.38,0.63,0.69}{\textbf{\textit{{#1}}}}}
    
    
    % Define a nice break command that doesn't care if a line doesn't already
    % exist.
    \def\br{\hspace*{\fill} \\* }
    % Math Jax compatability definitions
    \def\gt{>}
    \def\lt{<}
    % Document parameters
    \title{Week-13-XPath-B}
    
    
    

    % Pygments definitions
    
\makeatletter
\def\PY@reset{\let\PY@it=\relax \let\PY@bf=\relax%
    \let\PY@ul=\relax \let\PY@tc=\relax%
    \let\PY@bc=\relax \let\PY@ff=\relax}
\def\PY@tok#1{\csname PY@tok@#1\endcsname}
\def\PY@toks#1+{\ifx\relax#1\empty\else%
    \PY@tok{#1}\expandafter\PY@toks\fi}
\def\PY@do#1{\PY@bc{\PY@tc{\PY@ul{%
    \PY@it{\PY@bf{\PY@ff{#1}}}}}}}
\def\PY#1#2{\PY@reset\PY@toks#1+\relax+\PY@do{#2}}

\expandafter\def\csname PY@tok@w\endcsname{\def\PY@tc##1{\textcolor[rgb]{0.73,0.73,0.73}{##1}}}
\expandafter\def\csname PY@tok@c\endcsname{\let\PY@it=\textit\def\PY@tc##1{\textcolor[rgb]{0.25,0.50,0.50}{##1}}}
\expandafter\def\csname PY@tok@cp\endcsname{\def\PY@tc##1{\textcolor[rgb]{0.74,0.48,0.00}{##1}}}
\expandafter\def\csname PY@tok@k\endcsname{\let\PY@bf=\textbf\def\PY@tc##1{\textcolor[rgb]{0.00,0.50,0.00}{##1}}}
\expandafter\def\csname PY@tok@kp\endcsname{\def\PY@tc##1{\textcolor[rgb]{0.00,0.50,0.00}{##1}}}
\expandafter\def\csname PY@tok@kt\endcsname{\def\PY@tc##1{\textcolor[rgb]{0.69,0.00,0.25}{##1}}}
\expandafter\def\csname PY@tok@o\endcsname{\def\PY@tc##1{\textcolor[rgb]{0.40,0.40,0.40}{##1}}}
\expandafter\def\csname PY@tok@ow\endcsname{\let\PY@bf=\textbf\def\PY@tc##1{\textcolor[rgb]{0.67,0.13,1.00}{##1}}}
\expandafter\def\csname PY@tok@nb\endcsname{\def\PY@tc##1{\textcolor[rgb]{0.00,0.50,0.00}{##1}}}
\expandafter\def\csname PY@tok@nf\endcsname{\def\PY@tc##1{\textcolor[rgb]{0.00,0.00,1.00}{##1}}}
\expandafter\def\csname PY@tok@nc\endcsname{\let\PY@bf=\textbf\def\PY@tc##1{\textcolor[rgb]{0.00,0.00,1.00}{##1}}}
\expandafter\def\csname PY@tok@nn\endcsname{\let\PY@bf=\textbf\def\PY@tc##1{\textcolor[rgb]{0.00,0.00,1.00}{##1}}}
\expandafter\def\csname PY@tok@ne\endcsname{\let\PY@bf=\textbf\def\PY@tc##1{\textcolor[rgb]{0.82,0.25,0.23}{##1}}}
\expandafter\def\csname PY@tok@nv\endcsname{\def\PY@tc##1{\textcolor[rgb]{0.10,0.09,0.49}{##1}}}
\expandafter\def\csname PY@tok@no\endcsname{\def\PY@tc##1{\textcolor[rgb]{0.53,0.00,0.00}{##1}}}
\expandafter\def\csname PY@tok@nl\endcsname{\def\PY@tc##1{\textcolor[rgb]{0.63,0.63,0.00}{##1}}}
\expandafter\def\csname PY@tok@ni\endcsname{\let\PY@bf=\textbf\def\PY@tc##1{\textcolor[rgb]{0.60,0.60,0.60}{##1}}}
\expandafter\def\csname PY@tok@na\endcsname{\def\PY@tc##1{\textcolor[rgb]{0.49,0.56,0.16}{##1}}}
\expandafter\def\csname PY@tok@nt\endcsname{\let\PY@bf=\textbf\def\PY@tc##1{\textcolor[rgb]{0.00,0.50,0.00}{##1}}}
\expandafter\def\csname PY@tok@nd\endcsname{\def\PY@tc##1{\textcolor[rgb]{0.67,0.13,1.00}{##1}}}
\expandafter\def\csname PY@tok@s\endcsname{\def\PY@tc##1{\textcolor[rgb]{0.73,0.13,0.13}{##1}}}
\expandafter\def\csname PY@tok@sd\endcsname{\let\PY@it=\textit\def\PY@tc##1{\textcolor[rgb]{0.73,0.13,0.13}{##1}}}
\expandafter\def\csname PY@tok@si\endcsname{\let\PY@bf=\textbf\def\PY@tc##1{\textcolor[rgb]{0.73,0.40,0.53}{##1}}}
\expandafter\def\csname PY@tok@se\endcsname{\let\PY@bf=\textbf\def\PY@tc##1{\textcolor[rgb]{0.73,0.40,0.13}{##1}}}
\expandafter\def\csname PY@tok@sr\endcsname{\def\PY@tc##1{\textcolor[rgb]{0.73,0.40,0.53}{##1}}}
\expandafter\def\csname PY@tok@ss\endcsname{\def\PY@tc##1{\textcolor[rgb]{0.10,0.09,0.49}{##1}}}
\expandafter\def\csname PY@tok@sx\endcsname{\def\PY@tc##1{\textcolor[rgb]{0.00,0.50,0.00}{##1}}}
\expandafter\def\csname PY@tok@m\endcsname{\def\PY@tc##1{\textcolor[rgb]{0.40,0.40,0.40}{##1}}}
\expandafter\def\csname PY@tok@gh\endcsname{\let\PY@bf=\textbf\def\PY@tc##1{\textcolor[rgb]{0.00,0.00,0.50}{##1}}}
\expandafter\def\csname PY@tok@gu\endcsname{\let\PY@bf=\textbf\def\PY@tc##1{\textcolor[rgb]{0.50,0.00,0.50}{##1}}}
\expandafter\def\csname PY@tok@gd\endcsname{\def\PY@tc##1{\textcolor[rgb]{0.63,0.00,0.00}{##1}}}
\expandafter\def\csname PY@tok@gi\endcsname{\def\PY@tc##1{\textcolor[rgb]{0.00,0.63,0.00}{##1}}}
\expandafter\def\csname PY@tok@gr\endcsname{\def\PY@tc##1{\textcolor[rgb]{1.00,0.00,0.00}{##1}}}
\expandafter\def\csname PY@tok@ge\endcsname{\let\PY@it=\textit}
\expandafter\def\csname PY@tok@gs\endcsname{\let\PY@bf=\textbf}
\expandafter\def\csname PY@tok@gp\endcsname{\let\PY@bf=\textbf\def\PY@tc##1{\textcolor[rgb]{0.00,0.00,0.50}{##1}}}
\expandafter\def\csname PY@tok@go\endcsname{\def\PY@tc##1{\textcolor[rgb]{0.53,0.53,0.53}{##1}}}
\expandafter\def\csname PY@tok@gt\endcsname{\def\PY@tc##1{\textcolor[rgb]{0.00,0.27,0.87}{##1}}}
\expandafter\def\csname PY@tok@err\endcsname{\def\PY@bc##1{\setlength{\fboxsep}{0pt}\fcolorbox[rgb]{1.00,0.00,0.00}{1,1,1}{\strut ##1}}}
\expandafter\def\csname PY@tok@kc\endcsname{\let\PY@bf=\textbf\def\PY@tc##1{\textcolor[rgb]{0.00,0.50,0.00}{##1}}}
\expandafter\def\csname PY@tok@kd\endcsname{\let\PY@bf=\textbf\def\PY@tc##1{\textcolor[rgb]{0.00,0.50,0.00}{##1}}}
\expandafter\def\csname PY@tok@kn\endcsname{\let\PY@bf=\textbf\def\PY@tc##1{\textcolor[rgb]{0.00,0.50,0.00}{##1}}}
\expandafter\def\csname PY@tok@kr\endcsname{\let\PY@bf=\textbf\def\PY@tc##1{\textcolor[rgb]{0.00,0.50,0.00}{##1}}}
\expandafter\def\csname PY@tok@bp\endcsname{\def\PY@tc##1{\textcolor[rgb]{0.00,0.50,0.00}{##1}}}
\expandafter\def\csname PY@tok@fm\endcsname{\def\PY@tc##1{\textcolor[rgb]{0.00,0.00,1.00}{##1}}}
\expandafter\def\csname PY@tok@vc\endcsname{\def\PY@tc##1{\textcolor[rgb]{0.10,0.09,0.49}{##1}}}
\expandafter\def\csname PY@tok@vg\endcsname{\def\PY@tc##1{\textcolor[rgb]{0.10,0.09,0.49}{##1}}}
\expandafter\def\csname PY@tok@vi\endcsname{\def\PY@tc##1{\textcolor[rgb]{0.10,0.09,0.49}{##1}}}
\expandafter\def\csname PY@tok@vm\endcsname{\def\PY@tc##1{\textcolor[rgb]{0.10,0.09,0.49}{##1}}}
\expandafter\def\csname PY@tok@sa\endcsname{\def\PY@tc##1{\textcolor[rgb]{0.73,0.13,0.13}{##1}}}
\expandafter\def\csname PY@tok@sb\endcsname{\def\PY@tc##1{\textcolor[rgb]{0.73,0.13,0.13}{##1}}}
\expandafter\def\csname PY@tok@sc\endcsname{\def\PY@tc##1{\textcolor[rgb]{0.73,0.13,0.13}{##1}}}
\expandafter\def\csname PY@tok@dl\endcsname{\def\PY@tc##1{\textcolor[rgb]{0.73,0.13,0.13}{##1}}}
\expandafter\def\csname PY@tok@s2\endcsname{\def\PY@tc##1{\textcolor[rgb]{0.73,0.13,0.13}{##1}}}
\expandafter\def\csname PY@tok@sh\endcsname{\def\PY@tc##1{\textcolor[rgb]{0.73,0.13,0.13}{##1}}}
\expandafter\def\csname PY@tok@s1\endcsname{\def\PY@tc##1{\textcolor[rgb]{0.73,0.13,0.13}{##1}}}
\expandafter\def\csname PY@tok@mb\endcsname{\def\PY@tc##1{\textcolor[rgb]{0.40,0.40,0.40}{##1}}}
\expandafter\def\csname PY@tok@mf\endcsname{\def\PY@tc##1{\textcolor[rgb]{0.40,0.40,0.40}{##1}}}
\expandafter\def\csname PY@tok@mh\endcsname{\def\PY@tc##1{\textcolor[rgb]{0.40,0.40,0.40}{##1}}}
\expandafter\def\csname PY@tok@mi\endcsname{\def\PY@tc##1{\textcolor[rgb]{0.40,0.40,0.40}{##1}}}
\expandafter\def\csname PY@tok@il\endcsname{\def\PY@tc##1{\textcolor[rgb]{0.40,0.40,0.40}{##1}}}
\expandafter\def\csname PY@tok@mo\endcsname{\def\PY@tc##1{\textcolor[rgb]{0.40,0.40,0.40}{##1}}}
\expandafter\def\csname PY@tok@ch\endcsname{\let\PY@it=\textit\def\PY@tc##1{\textcolor[rgb]{0.25,0.50,0.50}{##1}}}
\expandafter\def\csname PY@tok@cm\endcsname{\let\PY@it=\textit\def\PY@tc##1{\textcolor[rgb]{0.25,0.50,0.50}{##1}}}
\expandafter\def\csname PY@tok@cpf\endcsname{\let\PY@it=\textit\def\PY@tc##1{\textcolor[rgb]{0.25,0.50,0.50}{##1}}}
\expandafter\def\csname PY@tok@c1\endcsname{\let\PY@it=\textit\def\PY@tc##1{\textcolor[rgb]{0.25,0.50,0.50}{##1}}}
\expandafter\def\csname PY@tok@cs\endcsname{\let\PY@it=\textit\def\PY@tc##1{\textcolor[rgb]{0.25,0.50,0.50}{##1}}}

\def\PYZbs{\char`\\}
\def\PYZus{\char`\_}
\def\PYZob{\char`\{}
\def\PYZcb{\char`\}}
\def\PYZca{\char`\^}
\def\PYZam{\char`\&}
\def\PYZlt{\char`\<}
\def\PYZgt{\char`\>}
\def\PYZsh{\char`\#}
\def\PYZpc{\char`\%}
\def\PYZdl{\char`\$}
\def\PYZhy{\char`\-}
\def\PYZsq{\char`\'}
\def\PYZdq{\char`\"}
\def\PYZti{\char`\~}
% for compatibility with earlier versions
\def\PYZat{@}
\def\PYZlb{[}
\def\PYZrb{]}
\makeatother


    % Exact colors from NB
    \definecolor{incolor}{rgb}{0.0, 0.0, 0.5}
    \definecolor{outcolor}{rgb}{0.545, 0.0, 0.0}



    
    % Prevent overflowing lines due to hard-to-break entities
    \sloppy 
    % Setup hyperref package
    \hypersetup{
      breaklinks=true,  % so long urls are correctly broken across lines
      colorlinks=true,
      urlcolor=urlcolor,
      linkcolor=linkcolor,
      citecolor=citecolor,
      }
    % Slightly bigger margins than the latex defaults
    
    \geometry{verbose,tmargin=1in,bmargin=1in,lmargin=1in,rmargin=1in}
    
    

    \begin{document}
    
    
    \maketitle
    
    

    
    \section{Spring 2018 Status: Looking
Ok.}\label{spring-2018-status-looking-ok.}

\section{XPath B}\label{xpath-b}

Now that you have gotten a bit more comfortable with XPath queries,
we're going to now explore how we can use a tool inside of Python to
execute those queries. There are many XPath tools out there, and there
will always be differences in how individual parsers will work. So
expect to need time to acclimate yourself if you are switching to a
different one. You will have to test things out and adapt accordingly.

For the purposes of our lesson here, we're going to use the XPath
function within lxml's etree module. This module works well and
consistently over the years, but you may find that other packages will
work similarly.

I'm going to set this all up for you, so you can use this pattern
without thinking too much about it. But I will try and explain some of
it as we go.

    \section{The basic pattern}\label{the-basic-pattern}

The basic pattern that we will be exploring here is this:

\begin{enumerate}
\def\labelenumi{\arabic{enumi}.}
\tightlist
\item
  Read in the document
\item
  Parse that object (either an IO object or string depending on your
  pattern) into a tree object.
\item
  Apply you desired XPath things to that tree object.
\end{enumerate}

There will be one and only one way you'll need to do this class, but
there are other methods out there that you'll see.

Our pattern for class will be:

\begin{enumerate}
\def\labelenumi{\arabic{enumi}.}
\tightlist
\item
  Read the XML file with \texttt{.read()} to read in the text as a big
  string.
\item
  Pass that string the parsing method \texttt{.fromstring()} to parse
  into a tree object. (don't forget that you'll have to import this
  module)
\item
  Use the \texttt{.xpath()} function on that tree object to execute
  xpath queries on it.
\end{enumerate}

\subsection{Step 0: import the lxml
module}\label{step-0-import-the-lxml-module}

\begin{Shaded}
\begin{Highlighting}[]
\ImportTok{from}\NormalTok{ lxml }\ImportTok{import}\NormalTok{ etree}
\end{Highlighting}
\end{Shaded}

\subsection{Step 1: read in the file}\label{step-1-read-in-the-file}

The \texttt{rb} read in mode is required because of the encoding issues.

\begin{Shaded}
\begin{Highlighting}[]
\NormalTok{infile }\OperatorTok{=}  \BuiltInTok{open}\NormalTok{(}\StringTok{'YOURFILENAME.xml'}\NormalTok{, }\StringTok{'rb'}\NormalTok{)}
\NormalTok{xml }\OperatorTok{=}\NormalTok{ infile.read() }\CommentTok{# this will be passed to the parser}
\NormalTok{infile.close()}
\end{Highlighting}
\end{Shaded}

\section{Step 2: parse into a tree
object}\label{step-2-parse-into-a-tree-object}

I could call this variable name anything that I want, but we usually use
\texttt{tree} as a convention to indicate that it is the entire XML tree
and not a constituent node. This is using the \texttt{.fromstring()}
function from the lxml/etree, which will parse string text into a tree
object.

\begin{Shaded}
\begin{Highlighting}[]
\NormalTok{tree }\OperatorTok{=}\NormalTok{ etree.fromstring(xml)}
\end{Highlighting}
\end{Shaded}

\section{\texorpdfstring{Step 3: use the \texttt{.xpath()} on the tree
object to execute an xpath
query}{Step 3: use the .xpath() on the tree object to execute an xpath query}}\label{step-3-use-the-.xpath-on-the-tree-object-to-execute-an-xpath-query}

We'll be talking about namespaces in a later section.

Example when there in no namespace happening:

\begin{Shaded}
\begin{Highlighting}[]
\NormalTok{results }\OperatorTok{=}\NormalTok{ tree.xpath(}\StringTok{'//elementwhatever/text()'}\NormalTok{)}
\end{Highlighting}
\end{Shaded}

Example when there is a namespace schema to handle:

\begin{Shaded}
\begin{Highlighting}[]
\NormalTok{results }\OperatorTok{=}\NormalTok{ tree.xpath(}\StringTok{'//alias:elementwhatever/text()'}\NormalTok{, namespaces}\OperatorTok{=}\NormalTok{\{}\StringTok{'alias'}\NormalTok{: }\StringTok{"URL found in the document goes here"}\NormalTok{\})}
\end{Highlighting}
\end{Shaded}

Don't worry, this entire lesson is about unpacking more about step 3.

\section{Our data source}\label{our-data-source}

We'll be using an XML document of "Hamlet" by Shakespeare. This is
located in the hamlet-tei.xml file. This is a proper XML file that uses
the TEI schema. https://en.wikipedia.org/wiki/Text\_Encoding\_Initiative
You will want to read this now so you can understand the basics of
what's going on in this file.

The data file has its own attribution, but I grabbed it as a material
from this workshop: http://tei.it.ox.ac.uk/Talks/2015-08-maynooth

Take some time exploring this file to better understand the structure.
There's no real need to do a full TEI tutorial for this lesson. This
lesson is not meant to be a tutorial on TEI, we're just using it as
example data.

This is a very brief description of the structure of the Hamlet file:

\begin{itemize}
\tightlist
\item
  In \texttt{teiHeader}:

  \begin{itemize}
  \tightlist
  \item
    \texttt{fileDesc} node contains information about the provenance of
    the file and content.
  \item
    \texttt{profileDesc/particDesc} node contains information on the
    characters in the play
  \item
    \texttt{profileDesc/settingDesc} node contains setting information
    for the play
  \end{itemize}
\item
  In \texttt{text}:

  \begin{itemize}
  \tightlist
  \item
    this contains nodes for each act, scene, and passage.
  \item
    each passage is in \texttt{sp} elements, with \texttt{@who}
    attributes representing the standardized ID for each speaker (those
    IDs are defined in the \texttt{particDesc} node. The
    \texttt{speaker} reports out what the original text had for the
    speaker information, and the \texttt{l} elements have the individual
    lines.
  \end{itemize}
\end{itemize}

There are other details that you will need to explore on you own to get
a feel for things.

For now, we're going to go ahead and read in our file and prepare our
tree object. You'll only need to do this once at the top of your script.
After that, you'll just be using the tree object.

    \begin{Verbatim}[commandchars=\\\{\}]
{\color{incolor}In [{\color{incolor}62}]:} \PY{k+kn}{from} \PY{n+nn}{lxml} \PY{k}{import} \PY{n}{etree}
         
         \PY{n}{infile} \PY{o}{=}  \PY{n+nb}{open}\PY{p}{(}\PY{l+s+s1}{\PYZsq{}}\PY{l+s+s1}{hamlet\PYZhy{}tei.xml}\PY{l+s+s1}{\PYZsq{}}\PY{p}{,} \PY{l+s+s1}{\PYZsq{}}\PY{l+s+s1}{rb}\PY{l+s+s1}{\PYZsq{}}\PY{p}{)} \PY{c+c1}{\PYZsh{} don\PYZsq{}t forget that rb in here}
         \PY{n}{xml} \PY{o}{=} \PY{n}{infile}\PY{o}{.}\PY{n}{read}\PY{p}{(}\PY{p}{)}
         \PY{n}{infile}\PY{o}{.}\PY{n}{close}\PY{p}{(}\PY{p}{)}
         
         \PY{n}{tree} \PY{o}{=} \PY{n}{etree}\PY{o}{.}\PY{n}{fromstring}\PY{p}{(}\PY{n}{xml}\PY{p}{)}
\end{Verbatim}


    \section{namespaces}\label{namespaces}

Most proper XML files have namespaces (there can be multiple) that
you'll need to navigate. As this is not a metadata or TEI course, I will
not provide an extended discussion on what this is.

We can see in line 4 of the document, which has the root element:
\texttt{\textless{}TEI\ xmlns="http://www.tei-c.org/ns/1.0"\textgreater{}}

This is saying that the elements found in this root node belong to the
TEI schema, with a URL to the schema definition. This information is for
the parsers that are trying to validate the schema to ensure that the
XML conforms to the schema. Programs such as Oxygen XML editor will do
this.

Sometimes it can be tricky to tell what the namespaces are for the
elements, but you'll see this URL pop up again when we dig in and print
out the elements. That is often the clue that you need for how to handle
the namespaces.

The patterns from step 3 show you how to handle this, but keep on
reading. Right now, I want you to focus on the larger picture and just
follow along with the pattern that I'm giving you. The purpose and usage
of the namespace will make more sense as you start using it. For now,
move on and use the pattern as I give it to you.

We'll be talking about that namespace a ton, so the canonical pattern is
to save that namespace dictionary as a variable that we can reference
elsewhere. We can same this now so we can reference it elsewhere.

The namespace dictionary can have multiple namespaces declared, where
the alias is the key (as a string) and the value is the URL (as a
string) as seen in the file. You may use any alias you would like for
that namespace, but the URL must perfectly match what appears in the
file.

Provide as many alias: URL pairs as you need for your document. Most
namespaces have a canonical alias to use, which you should abide by when
possible.

    \begin{Verbatim}[commandchars=\\\{\}]
{\color{incolor}In [{\color{incolor}63}]:} \PY{n}{ns} \PY{o}{=} \PY{p}{\PYZob{}}\PY{l+s+s1}{\PYZsq{}}\PY{l+s+s1}{tei}\PY{l+s+s1}{\PYZsq{}}\PY{p}{:} \PY{l+s+s1}{\PYZsq{}}\PY{l+s+s1}{http://www.tei\PYZhy{}c.org/ns/1.0}\PY{l+s+s1}{\PYZsq{}}\PY{p}{\PYZcb{}}
\end{Verbatim}


    \section{Evaluating an extraction query to get a single
result}\label{evaluating-an-extraction-query-to-get-a-single-result}

Remember that all your previous queries all needed to end with an
extraction function at the end. This was likely either \texttt{/text()}
to get the text of the element out, or \texttt{@attribute} to get some
attribute text out.

For example, \texttt{//a} would select all the \texttt{a} element nodes,
but not yield the contents. But \texttt{//a/text()} would give you the
hyperlink text, and \texttt{//a/@href} would give you all the URLs for
the hyperlinks.

The reasons for this aren't always made clear by those GUI tools we were
using. However, the distinction between a selection and extraction query
will be striking when using this tool in Python. You \textbf{must}
include an extraction statement in your query to get text content out.
Otherwise you'll be selecting elements, and nothing will appear useful
in the list of results.

As a start, we're going to run a query that will extract out a single
result. We're going to look up the standard name of Hamlet from his
character data node.

The xpath that we would want to use is
\texttt{//person{[}@xml:id\ =\ "F-ham-ham"{]}/persName{[}@type\ =\ "standard"{]}/text()},
but we need to adapt this to our namespace. Look back up to our
\texttt{ns} dictionary and look at what we declared the alias to be. We
gave our TEI namespace schema an alias of \texttt{tei}, which means we
need to provide this before each element name that we are referencing.
IMPORTANT! You only need to do this for element names, not for attribute
values, content, or XPath functions. Literally only for the element
names, but for \textbf{every} element name. Even when you have multiple.

So now our new XPath query with correct aliases will be:

\texttt{//tei:person{[}@xml:id\ =\ "F-ham-ham"{]}/tei:persName{[}@type\ =\ "standard"{]}/text()}

See those \texttt{tei:person} and \texttt{tei:persName}? That's how you
use that alias value. It's \texttt{alias:element}.

Let's put this together and see the results.

    \begin{Verbatim}[commandchars=\\\{\}]
{\color{incolor}In [{\color{incolor}64}]:} \PY{n+nb}{print}\PY{p}{(}\PY{n}{tree}\PY{o}{.}\PY{n}{xpath}\PY{p}{(}\PY{l+s+s1}{\PYZsq{}}\PY{l+s+s1}{//tei:person[@xml:id = }\PY{l+s+s1}{\PYZdq{}}\PY{l+s+s1}{F\PYZhy{}ham\PYZhy{}ham}\PY{l+s+s1}{\PYZdq{}}\PY{l+s+s1}{]/tei:persName[@type = }\PY{l+s+s1}{\PYZdq{}}\PY{l+s+s1}{standard}\PY{l+s+s1}{\PYZdq{}}\PY{l+s+s1}{]/text()}\PY{l+s+s1}{\PYZsq{}}\PY{p}{,} \PY{n}{namespaces} \PY{o}{=} \PY{n}{ns}\PY{p}{)}\PY{p}{)}
\end{Verbatim}


    \begin{Verbatim}[commandchars=\\\{\}]
['Hamlet, son of the former king and nephew to the\textbackslash{}n                            present king']

    \end{Verbatim}

    Things to note:

\begin{itemize}
\tightlist
\item
  I am using my alias here only for the elements, and that alias name
  matches what I have declared in my \texttt{ns} object.
\item
  I have \texttt{namespaces\ =\ ns} which will need to be in
  \textbf{each and every xpath query you run for this assignment}.
\item
  my xpath query is just a string
\item
  I've used double quotes in my xpath query, which means that I need to
  use single quotes to surround the string.
\item
  my results are coming back as a list with one element. I know and
  expect there to be just a single result, but the results will always
  be coming back to you as a list.
\item
  that extra text is from a the newline in the XML file itself.
\end{itemize}

    \section{Query to extract many
results}\label{query-to-extract-many-results}

Let's adapt our previous result to find all the standard names for these
characters. We don't need to change much. We need to take out the
\texttt{@xml:id\ =\ "F-ham-ham"} that selected only Hamlet's node, and
now it will select all the person nodes.

    \begin{Verbatim}[commandchars=\\\{\}]
{\color{incolor}In [{\color{incolor}65}]:} \PY{n}{results} \PY{o}{=} \PY{n}{tree}\PY{o}{.}\PY{n}{xpath}\PY{p}{(}\PY{l+s+s1}{\PYZsq{}}\PY{l+s+s1}{//tei:person/tei:persName[@type = }\PY{l+s+s1}{\PYZdq{}}\PY{l+s+s1}{standard}\PY{l+s+s1}{\PYZdq{}}\PY{l+s+s1}{]/text()}\PY{l+s+s1}{\PYZsq{}}\PY{p}{,} \PY{n}{namespaces} \PY{o}{=} \PY{n}{ns}\PY{p}{)}
         \PY{n+nb}{print}\PY{p}{(}\PY{n}{results}\PY{p}{)}
\end{Verbatim}


    \begin{Verbatim}[commandchars=\\\{\}]
['First Player', 'All', 'Ambassador', 'Player Prologue', 'Player Queen', 'Bernardo, sentinel', 'Norwegian Captain', 'First Clown', 'Fortinbras, Prince of ', 'Francisco, a soldier', 'Gentleman, courtier', 'Gentlemen', "Father's Ghost, Ghost of Hamlet's\textbackslash{}n                            Father", 'Guildenstern, courtier', 'Hamlet, son of the former king and nephew to the\textbackslash{}n                            present king', 'Horatio, friend to Hamlet', 'Claudius, King of Denmark', 'Laertes, son to Polonius', 'Lucianus', 'Marcellus, Officer', 'Messenger', 'Ophelia, daughter to Polonius', 'Osric, courtier', 'Second Clown', 'Polonius, Lord Chamberlain', 'Player King', 'Priest', 'Gertrude, Queen of Denmark and mother to\textbackslash{}n                            Hamlet', 'Rosencrantz, courtier', 'Reynaldo, servant to Polonius', 'Sailor', 'Servant', 'Voltemand, courtier']

    \end{Verbatim}

    Now we have a list of results to play with!

How many characters have standard names?

    \begin{Verbatim}[commandchars=\\\{\}]
{\color{incolor}In [{\color{incolor}66}]:} \PY{n+nb}{print}\PY{p}{(}\PY{n+nb}{len}\PY{p}{(}\PY{n}{results}\PY{p}{)}\PY{p}{)}
\end{Verbatim}


    \begin{Verbatim}[commandchars=\\\{\}]
33

    \end{Verbatim}

    Loop through the names and normalize the spaces.

    \begin{Verbatim}[commandchars=\\\{\}]
{\color{incolor}In [{\color{incolor}67}]:} \PY{k}{for} \PY{n}{name} \PY{o+ow}{in} \PY{n}{results}\PY{p}{:}
             \PY{n+nb}{print}\PY{p}{(}\PY{l+s+s2}{\PYZdq{}}\PY{l+s+s2}{ }\PY{l+s+s2}{\PYZdq{}}\PY{o}{.}\PY{n}{join}\PY{p}{(}\PY{n}{name}\PY{o}{.}\PY{n}{split}\PY{p}{(}\PY{p}{)}\PY{p}{)}\PY{p}{)}
\end{Verbatim}


    \begin{Verbatim}[commandchars=\\\{\}]
First Player
All
Ambassador
Player Prologue
Player Queen
Bernardo, sentinel
Norwegian Captain
First Clown
Fortinbras, Prince of
Francisco, a soldier
Gentleman, courtier
Gentlemen
Father's Ghost, Ghost of Hamlet's Father
Guildenstern, courtier
Hamlet, son of the former king and nephew to the present king
Horatio, friend to Hamlet
Claudius, King of Denmark
Laertes, son to Polonius
Lucianus
Marcellus, Officer
Messenger
Ophelia, daughter to Polonius
Osric, courtier
Second Clown
Polonius, Lord Chamberlain
Player King
Priest
Gertrude, Queen of Denmark and mother to Hamlet
Rosencrantz, courtier
Reynaldo, servant to Polonius
Sailor
Servant
Voltemand, courtier

    \end{Verbatim}

    \section{Profiling structures}\label{profiling-structures}

You can't be an expert in all schemas, so sometimes you need to use some
tools in python to profile the data that you are working with.

We can look inside the Hamlet person node and see that there are 4
reported variations:

\begin{Shaded}
\begin{Highlighting}[]
\KeywordTok{<persName}\OtherTok{ type=}\StringTok{"form"}\KeywordTok{>}\NormalTok{Ha.}\KeywordTok{</persName>}
\KeywordTok{<persName}\OtherTok{ type=}\StringTok{"form"}\KeywordTok{>}\NormalTok{Ham.}\KeywordTok{</persName>}
\KeywordTok{<persName}\OtherTok{ type=}\StringTok{"form"}\KeywordTok{>}\NormalTok{Hamlet.}\KeywordTok{</persName>}
\KeywordTok{<persName}\OtherTok{ type=}\StringTok{"form"}\KeywordTok{>}\NormalTok{Hem.}\KeywordTok{</persName>}
\end{Highlighting}
\end{Shaded}

But can we confirm that this really is the case? Alternatively, what if
we were the ones writing this data file and needed to fill this in?
Also, this doesn't include the counts, so we don't really know the
distribution of these forms.

Let's write a query that finds all the speaker representations of
Hamlet, and then runs the results through the couter tool that we've
seen before.

Here's our xpath to find all of Hamlet's passages:

\texttt{//tei:sp{[}@who\ =\ "\#F-ham-ham"{]}}

Now find all the speaker elements in there.

\texttt{//tei:sp{[}@who\ =\ "\#F-ham-ham"{]}/tei:speaker}

Now get all that text out!

\texttt{//tei:sp{[}@who\ =\ "\#F-ham-ham"{]}/tei:speaker/text()}

    \begin{Verbatim}[commandchars=\\\{\}]
{\color{incolor}In [{\color{incolor}68}]:} \PY{n}{results} \PY{o}{=} \PY{n}{tree}\PY{o}{.}\PY{n}{xpath}\PY{p}{(}\PY{l+s+s1}{\PYZsq{}}\PY{l+s+s1}{//tei:sp[@who = }\PY{l+s+s1}{\PYZdq{}}\PY{l+s+s1}{\PYZsh{}F\PYZhy{}ham\PYZhy{}ham}\PY{l+s+s1}{\PYZdq{}}\PY{l+s+s1}{]/tei:speaker/text()}\PY{l+s+s1}{\PYZsq{}}\PY{p}{,} \PY{n}{namespaces} \PY{o}{=} \PY{n}{ns}\PY{p}{)}
         \PY{n+nb}{print}\PY{p}{(}\PY{n}{results}\PY{p}{)}
         \PY{n+nb}{print}\PY{p}{(}\PY{n+nb}{len}\PY{p}{(}\PY{n}{results}\PY{p}{)}\PY{p}{)}
\end{Verbatim}


    \begin{Verbatim}[commandchars=\\\{\}]
['Ham.', 'Ham.', 'Ham.', 'Ham.', 'Ham.', 'Ham.', 'Ham.', 'Ham.', 'Ham.', 'Ham.', 'Ham.', 'Ham.', 'Ham.', 'Ham.', 'Ham.', 'Ham.', 'Ham.', 'Ham.', 'Ham.', 'Ham.', 'Ham.', 'Ham.', 'Ham.', 'Ham.', 'Ham.', 'Ham.', 'Ham.', 'Ham.', 'Ham.', 'Ham.', 'Ham.', 'Ham.', 'Ham.', 'Ham.', 'Ham.', 'Ham.', 'Ham.', 'Ham.', 'Ham.', 'Ham.', 'Ham.', 'Ham.', 'Ham.', 'Ham.', 'Ham.', 'Ham.', 'Ham.', 'Ham.', 'Ham.', 'Ham.', 'Ham.', 'Ham.', 'Ham.', 'Ham.', 'Ham.', 'Ham.', 'Ham.', 'Ham.', 'Ham.', 'Ham.', 'Ham.', 'Ham.', 'Ham.', 'Ham.', 'Ham.', 'Ham.', 'Ham.', 'Ham.', 'Ham.', 'Ham.', 'Ham.', 'Ham.', 'Ham.', 'Ham.', 'Ham.', 'Ham.', 'Ham.', 'Ham.', 'Ham.', 'Ham.', 'Ham.', 'Ham.', 'Ham.', 'Ham.', 'Ham.', 'Ham.', 'Ham.', 'Ham.', 'Ham.', 'Ham.', 'Ham.', 'Ham.', 'Ham.', 'Ham.', 'Ham.', 'Ham.', 'Ham.', 'Ham.', 'Ham.', 'Ham.', 'Ham.', 'Ham.', 'Ham.', 'Ham.', 'Ham.', 'Ham.', 'Ham.', 'Ham.', 'Ham.', 'Ham.', 'Ham.', 'Ham.', 'Ham.', 'Ham.', 'Ham.', 'Ham.', 'Ham.', 'Ham.', 'Ham.', 'Ham.', 'Ha.', 'Ham.', 'Ham.', 'Ham.', 'Ham.', 'Ham.', 'Ham.', 'Ham.', 'Ham.', 'Ham.', 'Ham.', 'Ham.', 'Ham.', 'Ham.', 'Ham.', 'Ham.', 'Ham.', 'Ham.', 'Ham.', 'Ham.', 'Ham.', 'Ham.', 'Ham.', 'Ham.', 'Ham.', 'Ham.', 'Ham.', 'Ham.', 'Ham.', 'Ham.', 'Ham.', 'Ham.', 'Ham.', 'Ham.', 'Ha.', 'Ham.', 'Ham.', 'Ham.', 'Ham.', 'Ham.', 'Ham.', 'Ham.', 'Ham.', 'Ham.', 'Ham.', 'Ham.', 'Ham.', 'Ham.', 'Ham.', 'Ham.', 'Ham.', 'Ham.', 'Ham.', 'Ham.', 'Ham.', 'Ham.', 'Ham.', 'Ham.', 'Ham.', 'Ham.', 'Ham.', 'Ham.', 'Ham.', 'Ham.', 'Ham.', 'Ham.', 'Ham.', 'Ham.', 'Ham.', 'Ham.', 'Ham.', 'Ham.', 'Ham.', 'Ham.', 'Ham.', 'Ham.', 'Ham.', 'Ham.', 'Ham.', 'Ham.', 'Ham.', 'Ham.', 'Ham.', 'Ham.', 'Ham.', 'Ham.', 'Ham.', 'Ham.', 'Ham.', 'Ham.', 'Ham.', 'Ham.', 'Ham.', 'Ham.', 'Ham.', 'Ham.', 'Ham.', 'Ham.', 'Ham.', 'Ham.', 'Ham.', 'Ham.', 'Ham.', 'Ham.', 'Ham.', 'Ham.', 'Ham.', 'Ham.', 'Ham.', 'Ham.', 'Ham.', 'Ham.', 'Ham.', 'Ham.', 'Ham.', 'Ham.', 'Ham.', 'Ham.', 'Ham.', 'Ham.', 'Ham.', 'Ham.', 'Ham.', 'Ham.', 'Ham.', 'Ham.', 'Ham.', 'Ham.', 'Ham.', 'Ham.', 'Ham.', 'Ham.', 'Hamlet.', 'Ham.', 'Ham.', 'Ham.', 'Ham.', 'Ham.', 'Ham.', 'Ham.', 'Ham.', 'Ham.', 'Ham.', 'Ham.', 'Ham.', 'Ham.', 'Ham.', 'Ham.', 'Ham.', 'Ham.', 'Ham.', 'Ham.', 'Ham.', 'Ham.', 'Ham.', 'Ham.', 'Ham.', 'Ham.', 'Ham.', 'Ham.', 'Ham.', 'Ham.', 'Ham.', 'Ham.', 'Ham.', 'Ham.', 'Ham.', 'Ham.', 'Ham.', 'Ham.', 'Ham.', 'Ham.', 'Ham.', 'Ham.', 'Ham.', 'Hem.', 'Ham.', 'Ham.', 'Ham.', 'Ham.', 'Ham.', 'Ham.', 'Ham.', 'Ham.', 'Ham.', 'Ham.', 'Ham.', 'Ham.', 'Ham.', 'Ham.', 'Ham.', 'Ham.', 'Ham.', 'Ham.', 'Ham.', 'Ham.', 'Ham.', 'Ham.', 'Ham.', 'Ham.', 'Ham.', 'Ham.', 'Ham.', 'Ham.', 'Ham.', 'Ham.', 'Ham.', 'Ham.', 'Ham.', 'Ham.', 'Ham.', 'Ham.', 'Ham.', 'Ham.', 'Ham.', 'Ham.', 'Ham.', 'Ham.', 'Ham.', 'Ham.']
340

    \end{Verbatim}

    There's 340 of these, which would be super annoying to count by hand.

    \begin{Verbatim}[commandchars=\\\{\}]
{\color{incolor}In [{\color{incolor}69}]:} \PY{k+kn}{from} \PY{n+nn}{collections} \PY{k}{import} \PY{n}{Counter}
         
         \PY{n+nb}{print}\PY{p}{(}\PY{n}{Counter}\PY{p}{(}\PY{n}{results}\PY{p}{)}\PY{p}{)}
\end{Verbatim}


    \begin{Verbatim}[commandchars=\\\{\}]
Counter(\{'Ham.': 336, 'Ha.': 2, 'Hamlet.': 1, 'Hem.': 1\})

    \end{Verbatim}

    So now we know more about this data! And in just a few lines of code.

    \section{Selecting nodes}\label{selecting-nodes}

Up to now, we've been focusing on the extraction of data. However, this
tool is much more powerful than that. As we've discussed with other data
structures in previous lectures, sometimes it can be really valuable to
isolate the specific data granularity that you want. Once you have those
chunks isolated, you can drill down into them to get out information
that you want. We can do the same thing here.

The value of being able to select just a node (instead of extracting
information out of it) is that you can save that object node as a
variable and apply xpath queries directly onto it. Yes, we could always
include that information in our original xpath if we were wanting a
single value. But sometimes we want more.

However, when we can isolate a node we can run however many xpath
queries we want on that node. And this is why it is powerful.

Some of the examples that we will be going through below could also be
done with xpath functions, but those aren't always supported inside
these packages. Also, this lesson is meant to highlight brining in data
into python.

So with that said, let's explore this.

You can easily select just the nodes for your query by omitting the
extraction chunk of your query.

We're curious about stage directions in speaker elements. These
directions have both text that we want and attribute values. We could
write this in two queries.

    \begin{Verbatim}[commandchars=\\\{\}]
{\color{incolor}In [{\color{incolor}70}]:} \PY{n}{stagedirtype} \PY{o}{=} \PY{n}{tree}\PY{o}{.}\PY{n}{xpath}\PY{p}{(}\PY{l+s+s1}{\PYZsq{}}\PY{l+s+s1}{//tei:sp/tei:stage/@type}\PY{l+s+s1}{\PYZsq{}}\PY{p}{,} \PY{n}{namespaces}\PY{o}{=}\PY{n}{ns}\PY{p}{)}
\end{Verbatim}


    \begin{Verbatim}[commandchars=\\\{\}]
{\color{incolor}In [{\color{incolor}71}]:} \PY{n}{stagedirtext} \PY{o}{=} \PY{n}{tree}\PY{o}{.}\PY{n}{xpath}\PY{p}{(}\PY{l+s+s1}{\PYZsq{}}\PY{l+s+s1}{//tei:sp/tei:stage/text()}\PY{l+s+s1}{\PYZsq{}}\PY{p}{,} \PY{n}{namespaces}\PY{o}{=}\PY{n}{ns}\PY{p}{)}
\end{Verbatim}


    \begin{Verbatim}[commandchars=\\\{\}]
{\color{incolor}In [{\color{incolor}72}]:} \PY{n+nb}{print}\PY{p}{(}\PY{n+nb}{len}\PY{p}{(}\PY{n}{stagedirtype}\PY{p}{)}\PY{p}{)}
         \PY{n+nb}{print}\PY{p}{(}\PY{n+nb}{len}\PY{p}{(}\PY{n}{stagedirtext}\PY{p}{)}\PY{p}{)}
\end{Verbatim}


    \begin{Verbatim}[commandchars=\\\{\}]
30
33

    \end{Verbatim}

    Hmmm, so if we did this with two queries, we can see that there are
differing length results. This means that the results don't line up via
positions, and there aren't ways that I can predict or know by just
looking at the content. So doing this as two separate passes won't work.

And indeed, there are some stage elements that do not have type
attributes. Example:
\texttt{\textless{}stage\ rend="italic\ inline"\textgreater{}within.\textless{}/stage\textgreater{}}

Using this structure of gathering all the elements and then extracting
the content allows us to navigate this kind of situation and provides
protection when we might not expect that to be the case.

Let's rewrite our query such that we only get stage elements that have
the type attribute (of any value). This time, we're only going to select
the matching elements, and not extract anything.

To check that an element contains an attribute value, we can place that
attribute reference in the logical check area, but with no operators. We
can select and gather all the elements by omitting any extraction
notations on the end.

    \begin{Verbatim}[commandchars=\\\{\}]
{\color{incolor}In [{\color{incolor}73}]:} \PY{n}{stagedirswithtype} \PY{o}{=} \PY{n}{tree}\PY{o}{.}\PY{n}{xpath}\PY{p}{(}\PY{l+s+s1}{\PYZsq{}}\PY{l+s+s1}{//tei:sp/tei:stage[@type]}\PY{l+s+s1}{\PYZsq{}}\PY{p}{,} \PY{n}{namespaces}\PY{o}{=}\PY{n}{ns}\PY{p}{)}
\end{Verbatim}


    \begin{Verbatim}[commandchars=\\\{\}]
{\color{incolor}In [{\color{incolor}74}]:} \PY{n}{stagedirswithtype}
\end{Verbatim}


\begin{Verbatim}[commandchars=\\\{\}]
{\color{outcolor}Out[{\color{outcolor}74}]:} [<Element \{http://www.tei-c.org/ns/1.0\}stage at 0x2e595199408>,
          <Element \{http://www.tei-c.org/ns/1.0\}stage at 0x2e595199108>,
          <Element \{http://www.tei-c.org/ns/1.0\}stage at 0x2e595103748>,
          <Element \{http://www.tei-c.org/ns/1.0\}stage at 0x2e5951907c8>,
          <Element \{http://www.tei-c.org/ns/1.0\}stage at 0x2e595199148>,
          <Element \{http://www.tei-c.org/ns/1.0\}stage at 0x2e595190248>,
          <Element \{http://www.tei-c.org/ns/1.0\}stage at 0x2e595190dc8>,
          <Element \{http://www.tei-c.org/ns/1.0\}stage at 0x2e595190148>,
          <Element \{http://www.tei-c.org/ns/1.0\}stage at 0x2e595190fc8>,
          <Element \{http://www.tei-c.org/ns/1.0\}stage at 0x2e595190208>,
          <Element \{http://www.tei-c.org/ns/1.0\}stage at 0x2e595190448>,
          <Element \{http://www.tei-c.org/ns/1.0\}stage at 0x2e595190508>,
          <Element \{http://www.tei-c.org/ns/1.0\}stage at 0x2e595190e88>,
          <Element \{http://www.tei-c.org/ns/1.0\}stage at 0x2e5951908c8>,
          <Element \{http://www.tei-c.org/ns/1.0\}stage at 0x2e595190088>,
          <Element \{http://www.tei-c.org/ns/1.0\}stage at 0x2e595185308>,
          <Element \{http://www.tei-c.org/ns/1.0\}stage at 0x2e595185508>,
          <Element \{http://www.tei-c.org/ns/1.0\}stage at 0x2e595190e48>,
          <Element \{http://www.tei-c.org/ns/1.0\}stage at 0x2e595185608>,
          <Element \{http://www.tei-c.org/ns/1.0\}stage at 0x2e595185348>,
          <Element \{http://www.tei-c.org/ns/1.0\}stage at 0x2e595185548>,
          <Element \{http://www.tei-c.org/ns/1.0\}stage at 0x2e595185b08>,
          <Element \{http://www.tei-c.org/ns/1.0\}stage at 0x2e595185b48>,
          <Element \{http://www.tei-c.org/ns/1.0\}stage at 0x2e5951851c8>,
          <Element \{http://www.tei-c.org/ns/1.0\}stage at 0x2e5951859c8>,
          <Element \{http://www.tei-c.org/ns/1.0\}stage at 0x2e5951853c8>,
          <Element \{http://www.tei-c.org/ns/1.0\}stage at 0x2e595185908>,
          <Element \{http://www.tei-c.org/ns/1.0\}stage at 0x2e595185c88>,
          <Element \{http://www.tei-c.org/ns/1.0\}stage at 0x2e5951858c8>,
          <Element \{http://www.tei-c.org/ns/1.0\}stage at 0x2e595185e48>]
\end{Verbatim}
            
    And when we print this out, we don't see text. We see that we are
storing objects in memory that have a nice method of printing (because
what would you print?). That's what that \textless{}\textgreater{} thing
means around them. We have Element objects stored, but we are getting
the default string representation.

While this might look like an error, it is exactly what we want!

We now have a list of objects, and we want to loop over them. Well, we
don't need anything fancy for that.

    \begin{Verbatim}[commandchars=\\\{\}]
{\color{incolor}In [{\color{incolor}75}]:} \PY{k}{for} \PY{n}{stageelem} \PY{o+ow}{in} \PY{n}{stagedirswithtype}\PY{p}{:}
             \PY{n+nb}{print}\PY{p}{(}\PY{n}{stageelem}\PY{p}{)}
\end{Verbatim}


    \begin{Verbatim}[commandchars=\\\{\}]
<Element \{http://www.tei-c.org/ns/1.0\}stage at 0x2e595199408>
<Element \{http://www.tei-c.org/ns/1.0\}stage at 0x2e595199108>
<Element \{http://www.tei-c.org/ns/1.0\}stage at 0x2e595103748>
<Element \{http://www.tei-c.org/ns/1.0\}stage at 0x2e5951907c8>
<Element \{http://www.tei-c.org/ns/1.0\}stage at 0x2e595199148>
<Element \{http://www.tei-c.org/ns/1.0\}stage at 0x2e595190248>
<Element \{http://www.tei-c.org/ns/1.0\}stage at 0x2e595190dc8>
<Element \{http://www.tei-c.org/ns/1.0\}stage at 0x2e595190148>
<Element \{http://www.tei-c.org/ns/1.0\}stage at 0x2e595190fc8>
<Element \{http://www.tei-c.org/ns/1.0\}stage at 0x2e595190208>
<Element \{http://www.tei-c.org/ns/1.0\}stage at 0x2e595190448>
<Element \{http://www.tei-c.org/ns/1.0\}stage at 0x2e595190508>
<Element \{http://www.tei-c.org/ns/1.0\}stage at 0x2e595190e88>
<Element \{http://www.tei-c.org/ns/1.0\}stage at 0x2e5951908c8>
<Element \{http://www.tei-c.org/ns/1.0\}stage at 0x2e595190088>
<Element \{http://www.tei-c.org/ns/1.0\}stage at 0x2e595185308>
<Element \{http://www.tei-c.org/ns/1.0\}stage at 0x2e595185508>
<Element \{http://www.tei-c.org/ns/1.0\}stage at 0x2e595190e48>
<Element \{http://www.tei-c.org/ns/1.0\}stage at 0x2e595185608>
<Element \{http://www.tei-c.org/ns/1.0\}stage at 0x2e595185348>
<Element \{http://www.tei-c.org/ns/1.0\}stage at 0x2e595185548>
<Element \{http://www.tei-c.org/ns/1.0\}stage at 0x2e595185b08>
<Element \{http://www.tei-c.org/ns/1.0\}stage at 0x2e595185b48>
<Element \{http://www.tei-c.org/ns/1.0\}stage at 0x2e5951851c8>
<Element \{http://www.tei-c.org/ns/1.0\}stage at 0x2e5951859c8>
<Element \{http://www.tei-c.org/ns/1.0\}stage at 0x2e5951853c8>
<Element \{http://www.tei-c.org/ns/1.0\}stage at 0x2e595185908>
<Element \{http://www.tei-c.org/ns/1.0\}stage at 0x2e595185c88>
<Element \{http://www.tei-c.org/ns/1.0\}stage at 0x2e5951858c8>
<Element \{http://www.tei-c.org/ns/1.0\}stage at 0x2e595185e48>

    \end{Verbatim}

    Just because they don't have a pretty print out doesn't mean that we're
doing something wrong. These objects all represent the stage elements
that we selected, and can have xpath queries run directly on them.

Pretty much all your previous xpath query stuyy will work on them as
normal, with one exception: there is no root to handle here. So whatever
you run, it will be relevant to that element that you have selected.
Operationalizing this, it means that your xpath exressions don't need to
start with / or //, and effectively only need to be the piece you would
add on to an existing expression if you hadn't already selected them.

Let's take an HTML example here.

If I selected all a elements within a table with \texttt{//table//a},
but what I really wanted was the \texttt{href} attribute content, then
my xpath query would only be \texttt{@href} because the element is
already selected (as in, it's the object that I'm already iterating
over).

So now that we have our speaker elements, we want to loop over them to
collect multiple pieces of information about them. In this case, we want
the speaker ID, what type of direction it is, and then what the text of
the direction says.

We'll build this up. The easiest is adding the \texttt{@type} extraction
onto this.

    \begin{Verbatim}[commandchars=\\\{\}]
{\color{incolor}In [{\color{incolor}76}]:} \PY{k}{for} \PY{n}{stage} \PY{o+ow}{in} \PY{n}{stagedirswithtype}\PY{p}{:}
             \PY{n+nb}{print}\PY{p}{(}\PY{n}{stage}\PY{o}{.}\PY{n}{xpath}\PY{p}{(}\PY{l+s+s1}{\PYZsq{}}\PY{l+s+s1}{@type}\PY{l+s+s1}{\PYZsq{}}\PY{p}{,} \PY{n}{namespaces} \PY{o}{=} \PY{n}{ns}\PY{p}{)}\PY{p}{)}
\end{Verbatim}


    \begin{Verbatim}[commandchars=\\\{\}]
['entrance']
['entrance']
['exit']
['entrance']
['exit']
['entrance']
['business']
['business']
['exit']
['entrance']
['exit']
['entrance']
['exit']
['business']
['entrance']
['entrance']
['entrance']
['exit']
['location']
['entrance']
['business']
['business']
['entrance']
['exit']
['business']
['business']
['business']
['entrance']
['business']
['business']

    \end{Verbatim}

    \begin{Verbatim}[commandchars=\\\{\}]
{\color{incolor}In [{\color{incolor}77}]:} \PY{k}{for} \PY{n}{stage} \PY{o+ow}{in} \PY{n}{stagedirswithtype}\PY{p}{:}
             \PY{n+nb}{print}\PY{p}{(}\PY{n}{stage}\PY{o}{.}\PY{n}{xpath}\PY{p}{(}\PY{l+s+s1}{\PYZsq{}}\PY{l+s+s1}{@type}\PY{l+s+s1}{\PYZsq{}}\PY{p}{,} \PY{n}{namespaces} \PY{o}{=} \PY{n}{ns}\PY{p}{)}\PY{p}{,} \PY{n}{stage}\PY{o}{.}\PY{n}{xpath}\PY{p}{(}\PY{l+s+s1}{\PYZsq{}}\PY{l+s+s1}{text()}\PY{l+s+s1}{\PYZsq{}}\PY{p}{,} \PY{n}{namespaces} \PY{o}{=} \PY{n}{ns}\PY{p}{)}\PY{p}{)}
\end{Verbatim}


    \begin{Verbatim}[commandchars=\\\{\}]
['entrance'] ['Enter the\textbackslash{}n                                Ghost.']
['entrance'] ['Enter Ghost againe.']
['exit'] ['Exit Ghost.']
['entrance'] ['Enter Voltemand and\textbackslash{}n                                Cornelius.']
['exit'] ['Exit Voltemand and\textbackslash{}n                                Cornelius.']
['entrance'] ['Enter Polonius.']
['business'] ['within.']
['business'] ['The Letter.']
['exit'] ['Exit King \&\textbackslash{}n                                Queen.']
['entrance'] ['Enter foure or fiue\textbackslash{}n                                Players.']
['exit'] ['Exit Players.']
['entrance'] ['Enter Polonius, Rosincrance,\textbackslash{}n                                and Guildensterne.']
['exit'] ['Exit Polonius.']
['business'] ['Sleepes']
['entrance'] ['Enter Lucianus.']
['entrance'] ['Enter one with a\textbackslash{}n                                Recorder.']
['entrance'] ['Enter Ros. \&\textbackslash{}n                                Guild.']
['exit'] ['Exit Gent.']
['location'] ['within.']
['entrance'] ['Enter ', '.']
['business'] ['A noise\textbackslash{}n                                within.']
['business'] ['Reads the Letter.']
['entrance'] ['Enter a Messenger.']
['exit'] ['Exit Messenger']
['business'] ['Sings.']
['business'] ['sings.']
['business'] ['sings.']
['entrance'] ['Enter King, Queen, Laertes,\textbackslash{}n                                and a Coffin, ', 'with Lords attendant.']
['business'] ['Leaps in the\textbackslash{}n                                graue.']
['business'] ['March afarre off,\textbackslash{}n                                and shout within.']

    \end{Verbatim}

    We've got our two results now, but getting the person's ID might be
trickier. Well, not really, but it might look weird. Remember that we
have selected the stage elements, but the @who attribute is within the
parent sp element that we didn't select.

That's the wonderful thing about this, these are element objects, and
they know all about their parent elements. We can use .. as the
beginning of our query, just as if we were using it inside a longer
xpath expression.

    \begin{Verbatim}[commandchars=\\\{\}]
{\color{incolor}In [{\color{incolor}78}]:} \PY{k}{for} \PY{n}{stage} \PY{o+ow}{in} \PY{n}{stagedirswithtype}\PY{p}{:}
             \PY{n+nb}{print}\PY{p}{(}\PY{n}{stage}\PY{o}{.}\PY{n}{xpath}\PY{p}{(}\PY{l+s+s1}{\PYZsq{}}\PY{l+s+s1}{../@who}\PY{l+s+s1}{\PYZsq{}}\PY{p}{,} \PY{n}{namespaces} \PY{o}{=} \PY{n}{ns}\PY{p}{)}\PY{p}{)}
             \PY{n+nb}{print}\PY{p}{(}\PY{n}{stage}\PY{o}{.}\PY{n}{xpath}\PY{p}{(}\PY{l+s+s1}{\PYZsq{}}\PY{l+s+s1}{@type}\PY{l+s+s1}{\PYZsq{}}\PY{p}{,} \PY{n}{namespaces} \PY{o}{=} \PY{n}{ns}\PY{p}{)}\PY{p}{,} \PY{n}{stage}\PY{o}{.}\PY{n}{xpath}\PY{p}{(}\PY{l+s+s1}{\PYZsq{}}\PY{l+s+s1}{text()}\PY{l+s+s1}{\PYZsq{}}\PY{p}{,} \PY{n}{namespaces} \PY{o}{=} \PY{n}{ns}\PY{p}{)}\PY{p}{)}
\end{Verbatim}


    \begin{Verbatim}[commandchars=\\\{\}]
['\#F-ham-mar']
['entrance'] ['Enter the\textbackslash{}n                                Ghost.']
['\#F-ham-hor']
['entrance'] ['Enter Ghost againe.']
['\#F-ham-mar']
['exit'] ['Exit Ghost.']
['\#F-ham-cla']
['entrance'] ['Enter Voltemand and\textbackslash{}n                                Cornelius.']
['\#F-ham-cla']
['exit'] ['Exit Voltemand and\textbackslash{}n                                Cornelius.']
['\#F-ham-lae']
['entrance'] ['Enter Polonius.']
['\#F-ham-hor \#F-ham-mar']
['business'] ['within.']
['\#F-ham-pol']
['business'] ['The Letter.']
['\#F-ham-pol']
['exit'] ['Exit King \&\textbackslash{}n                                Queen.']
['\#F-ham-ham']
['entrance'] ['Enter foure or fiue\textbackslash{}n                                Players.']
['\#F-ham-ham']
['exit'] ['Exit Players.']
['\#F-ham-ham']
['entrance'] ['Enter Polonius, Rosincrance,\textbackslash{}n                                and Guildensterne.']
['\#F-ham-ham']
['exit'] ['Exit Polonius.']
['\#F-ham-ger']
['business'] ['Sleepes']
['\#F-ham-ham']
['entrance'] ['Enter Lucianus.']
['\#F-ham-ham']
['entrance'] ['Enter one with a\textbackslash{}n                                Recorder.']
['\#F-ham-cla']
['entrance'] ['Enter Ros. \&\textbackslash{}n                                Guild.']
['\#F-ham-cla']
['exit'] ['Exit Gent.']
['\#F-ham-gmn']
['location'] ['within.']
['\#F-ham-cla']
['entrance'] ['Enter ', '.']
['\#F-ham-cla']
['business'] ['A noise\textbackslash{}n                                within.']
['\#F-ham-sai']
['business'] ['Reads the Letter.']
['\#F-ham-cla']
['entrance'] ['Enter a Messenger.']
['\#F-ham-cla']
['exit'] ['Exit Messenger']
['\#F-ham-clo.1']
['business'] ['Sings.']
['\#F-ham-clo.1']
['business'] ['sings.']
['\#F-ham-clo.1']
['business'] ['sings.']
['\#F-ham-ham']
['entrance'] ['Enter King, Queen, Laertes,\textbackslash{}n                                and a Coffin, ', 'with Lords attendant.']
['\#F-ham-lae']
['business'] ['Leaps in the\textbackslash{}n                                graue.']
['\#F-ham-ham']
['business'] ['March afarre off,\textbackslash{}n                                and shout within.']

    \end{Verbatim}

    \section{making the results pretty}\label{making-the-results-pretty}

Now we've got a host of results, but the print is really ugly. We can
collect everything that we want into a single list for nice printing.
Remember that everything is inside of a list, so we need to extract out
the results.

    \begin{Verbatim}[commandchars=\\\{\}]
{\color{incolor}In [{\color{incolor}79}]:} \PY{k}{for} \PY{n}{stage} \PY{o+ow}{in} \PY{n}{stagedirswithtype}\PY{p}{:}
             \PY{n}{results} \PY{o}{=} \PY{p}{[}\PY{p}{]}
             \PY{n}{who} \PY{o}{=} \PY{n}{stage}\PY{o}{.}\PY{n}{xpath}\PY{p}{(}\PY{l+s+s1}{\PYZsq{}}\PY{l+s+s1}{../@who}\PY{l+s+s1}{\PYZsq{}}\PY{p}{,} \PY{n}{namespaces} \PY{o}{=} \PY{n}{ns}\PY{p}{)}
             \PY{n}{dirtype} \PY{o}{=} \PY{n}{stage}\PY{o}{.}\PY{n}{xpath}\PY{p}{(}\PY{l+s+s1}{\PYZsq{}}\PY{l+s+s1}{@type}\PY{l+s+s1}{\PYZsq{}}\PY{p}{,} \PY{n}{namespaces} \PY{o}{=} \PY{n}{ns}\PY{p}{)}
             \PY{n}{dirtext} \PY{o}{=} \PY{n}{stage}\PY{o}{.}\PY{n}{xpath}\PY{p}{(}\PY{l+s+s1}{\PYZsq{}}\PY{l+s+s1}{text()}\PY{l+s+s1}{\PYZsq{}}\PY{p}{,} \PY{n}{namespaces} \PY{o}{=} \PY{n}{ns}\PY{p}{)}
             \PY{n}{results}\PY{o}{.}\PY{n}{append}\PY{p}{(}\PY{n}{who}\PY{p}{[}\PY{l+m+mi}{0}\PY{p}{]}\PY{p}{)}
             \PY{n}{results}\PY{o}{.}\PY{n}{append}\PY{p}{(}\PY{n}{dirtype}\PY{p}{[}\PY{l+m+mi}{0}\PY{p}{]}\PY{p}{)}
             \PY{n}{results}\PY{o}{.}\PY{n}{append}\PY{p}{(}\PY{l+s+s2}{\PYZdq{}}\PY{l+s+s2}{ }\PY{l+s+s2}{\PYZdq{}}\PY{o}{.}\PY{n}{join}\PY{p}{(}\PY{n}{dirtext}\PY{p}{[}\PY{l+m+mi}{0}\PY{p}{]}\PY{o}{.}\PY{n}{split}\PY{p}{(}\PY{p}{)}\PY{p}{)}\PY{p}{)}
             \PY{n+nb}{print}\PY{p}{(}\PY{n}{results}\PY{p}{)}
\end{Verbatim}


    \begin{Verbatim}[commandchars=\\\{\}]
['\#F-ham-mar', 'entrance', 'Enter the Ghost.']
['\#F-ham-hor', 'entrance', 'Enter Ghost againe.']
['\#F-ham-mar', 'exit', 'Exit Ghost.']
['\#F-ham-cla', 'entrance', 'Enter Voltemand and Cornelius.']
['\#F-ham-cla', 'exit', 'Exit Voltemand and Cornelius.']
['\#F-ham-lae', 'entrance', 'Enter Polonius.']
['\#F-ham-hor \#F-ham-mar', 'business', 'within.']
['\#F-ham-pol', 'business', 'The Letter.']
['\#F-ham-pol', 'exit', 'Exit King \& Queen.']
['\#F-ham-ham', 'entrance', 'Enter foure or fiue Players.']
['\#F-ham-ham', 'exit', 'Exit Players.']
['\#F-ham-ham', 'entrance', 'Enter Polonius, Rosincrance, and Guildensterne.']
['\#F-ham-ham', 'exit', 'Exit Polonius.']
['\#F-ham-ger', 'business', 'Sleepes']
['\#F-ham-ham', 'entrance', 'Enter Lucianus.']
['\#F-ham-ham', 'entrance', 'Enter one with a Recorder.']
['\#F-ham-cla', 'entrance', 'Enter Ros. \& Guild.']
['\#F-ham-cla', 'exit', 'Exit Gent.']
['\#F-ham-gmn', 'location', 'within.']
['\#F-ham-cla', 'entrance', 'Enter']
['\#F-ham-cla', 'business', 'A noise within.']
['\#F-ham-sai', 'business', 'Reads the Letter.']
['\#F-ham-cla', 'entrance', 'Enter a Messenger.']
['\#F-ham-cla', 'exit', 'Exit Messenger']
['\#F-ham-clo.1', 'business', 'Sings.']
['\#F-ham-clo.1', 'business', 'sings.']
['\#F-ham-clo.1', 'business', 'sings.']
['\#F-ham-ham', 'entrance', 'Enter King, Queen, Laertes, and a Coffin,']
['\#F-ham-lae', 'business', 'Leaps in the graue.']
['\#F-ham-ham', 'business', 'March afarre off, and shout within.']

    \end{Verbatim}

    This has all worked nicely because everything had at least one value, so
we can hard code that {[}0{]} pretty safely. But this will not always we
the case. We can look up and see that one of the items has two speakers,
and the XML authors chose to handle that by listing them both in the
@who as "\#F-ham-hor \#F-ham-mar". So there's still only 1 result in the
sense that there's only one thing coming back to us with our query
(because you can't repeat attributes in a single element). So they added
both results to a single string and made that the single attribute's
value.

\section{Handling empty results}\label{handling-empty-results}

This is fine for now. We would need to accommodate our data model for
this if we were really doing research here.

However, we could take a look back to our original 33 results. Not all
of these had a type, right? So what happens if our query has no results?

You get an empty list.

    \begin{Verbatim}[commandchars=\\\{\}]
{\color{incolor}In [{\color{incolor}80}]:} \PY{n+nb}{print}\PY{p}{(}\PY{n}{tree}\PY{o}{.}\PY{n}{xpath}\PY{p}{(}\PY{l+s+s1}{\PYZsq{}}\PY{l+s+s1}{tei:frogs/text()}\PY{l+s+s1}{\PYZsq{}}\PY{p}{,} \PY{n}{namespaces}\PY{o}{=}\PY{n}{ns}\PY{p}{)}\PY{p}{)}
\end{Verbatim}


    \begin{Verbatim}[commandchars=\\\{\}]
[]

    \end{Verbatim}

    How you want to handle this really depends on what you are after and
your own business rules for the data that you are after. Usually you can
put in a little decision structure to check the length (or content if
you need). This is where your knowledge of the data will be very
important.

Let's use a function for checking that out results have only one item.
Otherwise, it will raise an error if it gets more, and fill in a missing
value if there is nothing.

Now, this function can't tell you WHY there is nothing, so again, know
your data and test your code.

Let's explore this function that I've prepared for you. This actually
has nothing to do with xpath at all, as the results that we get back
will be purely as a list.

    \begin{Verbatim}[commandchars=\\\{\}]
{\color{incolor}In [{\color{incolor}81}]:} \PY{k}{def} \PY{n+nf}{checkFor1Result}\PY{p}{(}\PY{n}{xpathresult}\PY{p}{,} \PY{n}{missing\PYZus{}value}\PY{p}{)}\PY{p}{:}
             \PY{k}{if} \PY{n+nb}{len}\PY{p}{(}\PY{n}{xpathresult}\PY{p}{)} \PY{o}{\PYZgt{}} \PY{l+m+mi}{1}\PY{p}{:} \PY{c+c1}{\PYZsh{} raise error if there are more than 1}
                 \PY{n}{howmany} \PY{o}{=} \PY{n+nb}{len}\PY{p}{(}\PY{n}{xpathresult}\PY{p}{)}
                 \PY{k}{raise} \PY{n+ne}{ValueError}\PY{p}{(}\PY{l+s+s2}{\PYZdq{}}\PY{l+s+s2}{Your list had }\PY{l+s+s2}{\PYZdq{}} \PY{o}{+} \PY{n+nb}{str}\PY{p}{(}\PY{n}{howmany}\PY{p}{)} \PY{o}{+} \PY{l+s+s2}{\PYZdq{}}\PY{l+s+s2}{ items instead of 1. Shutting down the program,}\PY{l+s+s2}{\PYZdq{}}
                                  \PY{o}{+} \PY{l+s+s2}{\PYZdq{}}\PY{l+s+s2}{But here}\PY{l+s+s2}{\PYZsq{}}\PY{l+s+s2}{s your failed result: }\PY{l+s+s2}{\PYZdq{}} \PY{o}{+} \PY{n+nb}{str}\PY{p}{(}\PY{n}{xpathresult}\PY{p}{)}\PY{p}{)}
             \PY{k}{elif} \PY{n+nb}{len}\PY{p}{(}\PY{n}{xpathresult}\PY{p}{)} \PY{o}{==} \PY{l+m+mi}{1}\PY{p}{:} 
                 \PY{n}{result} \PY{o}{=} \PY{n}{xpathresult}\PY{p}{[}\PY{l+m+mi}{0}\PY{p}{]} \PY{c+c1}{\PYZsh{} send the single result back when there\PYZsq{}s just 1}
             \PY{k}{else}\PY{p}{:} \PY{c+c1}{\PYZsh{} send the missing value back when there aren\PYZsq{}t any results}
                 \PY{n}{result} \PY{o}{=} \PY{n}{missing\PYZus{}value}
             \PY{k}{return} \PY{n}{result}
\end{Verbatim}


    \begin{Verbatim}[commandchars=\\\{\}]
{\color{incolor}In [{\color{incolor}82}]:} \PY{c+c1}{\PYZsh{} raises an error and halts the program if there are more than 1}
         \PY{n+nb}{print}\PY{p}{(}\PY{n}{checkFor1Result}\PY{p}{(}\PY{p}{[}\PY{l+s+s1}{\PYZsq{}}\PY{l+s+s1}{too}\PY{l+s+s1}{\PYZsq{}}\PY{p}{,} \PY{l+s+s1}{\PYZsq{}}\PY{l+s+s1}{many}\PY{l+s+s1}{\PYZsq{}}\PY{p}{]}\PY{p}{,} \PY{l+s+s1}{\PYZsq{}}\PY{l+s+s1}{MissingResult}\PY{l+s+s1}{\PYZsq{}}\PY{p}{)}\PY{p}{)}
\end{Verbatim}


    \begin{Verbatim}[commandchars=\\\{\}]

        ---------------------------------------------------------------------------

        ValueError                                Traceback (most recent call last)

        <ipython-input-82-db2a8f58fa3a> in <module>()
          1 \# raises an error and halts the program if there are more than 1
    ----> 2 print(checkFor1Result(['too', 'many'], 'MissingResult'))
    

        <ipython-input-81-68c835c76142> in checkFor1Result(xpathresult, missing\_value)
          3         howmany = len(xpathresult)
          4         raise ValueError("Your list had " + str(howmany) + " items instead of 1. Shutting down the program,"
    ----> 5                          + "But here's your failed result: " + str(xpathresult))
          6     elif len(xpathresult) == 1:
          7         result = xpathresult[0] \# send the single result back when there's just 1
    

        ValueError: Your list had 2 items instead of 1. Shutting down the program,But here's your failed result: ['too', 'many']

    \end{Verbatim}

    \begin{Verbatim}[commandchars=\\\{\}]
{\color{incolor}In [{\color{incolor}83}]:} \PY{c+c1}{\PYZsh{} passes the value through if there are exactly 1}
         \PY{n+nb}{print}\PY{p}{(}\PY{n}{checkFor1Result}\PY{p}{(}\PY{p}{[}\PY{l+s+s1}{\PYZsq{}}\PY{l+s+s1}{yup}\PY{l+s+s1}{\PYZsq{}}\PY{p}{]}\PY{p}{,} \PY{l+s+s1}{\PYZsq{}}\PY{l+s+s1}{MissingResult}\PY{l+s+s1}{\PYZsq{}}\PY{p}{)}\PY{p}{)}
\end{Verbatim}


    \begin{Verbatim}[commandchars=\\\{\}]
yup

    \end{Verbatim}

    \begin{Verbatim}[commandchars=\\\{\}]
{\color{incolor}In [{\color{incolor}84}]:} \PY{c+c1}{\PYZsh{} passes the missing result through if there\PYZsq{}s nothing.}
         \PY{n+nb}{print}\PY{p}{(}\PY{n}{checkFor1Result}\PY{p}{(}\PY{p}{[}\PY{p}{]}\PY{p}{,} \PY{l+s+s1}{\PYZsq{}}\PY{l+s+s1}{MissingResult}\PY{l+s+s1}{\PYZsq{}}\PY{p}{)}\PY{p}{)}
\end{Verbatim}


    \begin{Verbatim}[commandchars=\\\{\}]
MissingResult

    \end{Verbatim}

    This function guarentees that you will have 1 and only 1 result coming
back to you.

Let's see this in action on all 33 stage elements.

    \begin{Verbatim}[commandchars=\\\{\}]
{\color{incolor}In [{\color{incolor}85}]:} \PY{n}{allstages} \PY{o}{=} \PY{n}{tree}\PY{o}{.}\PY{n}{xpath}\PY{p}{(}\PY{l+s+s1}{\PYZsq{}}\PY{l+s+s1}{//tei:sp/tei:stage}\PY{l+s+s1}{\PYZsq{}}\PY{p}{,} \PY{n}{namespaces}\PY{o}{=}\PY{n}{ns}\PY{p}{)}
\end{Verbatim}


    \begin{Verbatim}[commandchars=\\\{\}]
{\color{incolor}In [{\color{incolor}86}]:} \PY{k}{for} \PY{n}{stage} \PY{o+ow}{in} \PY{n}{allstages}\PY{p}{:}
             \PY{n}{results} \PY{o}{=} \PY{p}{[}\PY{p}{]}
             \PY{n}{who} \PY{o}{=} \PY{n}{stage}\PY{o}{.}\PY{n}{xpath}\PY{p}{(}\PY{l+s+s1}{\PYZsq{}}\PY{l+s+s1}{../@who}\PY{l+s+s1}{\PYZsq{}}\PY{p}{,} \PY{n}{namespaces} \PY{o}{=} \PY{n}{ns}\PY{p}{)}
             \PY{n}{dirtype} \PY{o}{=} \PY{n}{stage}\PY{o}{.}\PY{n}{xpath}\PY{p}{(}\PY{l+s+s1}{\PYZsq{}}\PY{l+s+s1}{@type}\PY{l+s+s1}{\PYZsq{}}\PY{p}{,} \PY{n}{namespaces} \PY{o}{=} \PY{n}{ns}\PY{p}{)}
             \PY{n}{dirtext} \PY{o}{=} \PY{n}{stage}\PY{o}{.}\PY{n}{xpath}\PY{p}{(}\PY{l+s+s1}{\PYZsq{}}\PY{l+s+s1}{.//text()}\PY{l+s+s1}{\PYZsq{}}\PY{p}{,} \PY{n}{namespaces} \PY{o}{=} \PY{n}{ns}\PY{p}{)}
             \PY{n}{results}\PY{o}{.}\PY{n}{append}\PY{p}{(}\PY{n}{checkFor1Result}\PY{p}{(}\PY{n}{who}\PY{p}{,} \PY{l+s+s1}{\PYZsq{}}\PY{l+s+s1}{MissingWho}\PY{l+s+s1}{\PYZsq{}}\PY{p}{)}\PY{p}{)}
             \PY{n}{results}\PY{o}{.}\PY{n}{append}\PY{p}{(}\PY{n}{checkFor1Result}\PY{p}{(}\PY{n}{dirtype}\PY{p}{,} \PY{l+s+s1}{\PYZsq{}}\PY{l+s+s1}{MissingType}\PY{l+s+s1}{\PYZsq{}}\PY{p}{)}\PY{p}{)}
             \PY{n}{results}\PY{o}{.}\PY{n}{append}\PY{p}{(}\PY{l+s+s2}{\PYZdq{}}\PY{l+s+s2}{ }\PY{l+s+s2}{\PYZdq{}}\PY{o}{.}\PY{n}{join}\PY{p}{(}\PY{n}{checkFor1Result}\PY{p}{(}\PY{n}{dirtext}\PY{p}{,} \PY{l+s+s1}{\PYZsq{}}\PY{l+s+s1}{MissingText}\PY{l+s+s1}{\PYZsq{}}\PY{p}{)}\PY{o}{.}\PY{n}{split}\PY{p}{(}\PY{p}{)}\PY{p}{)}\PY{p}{)}
             \PY{n+nb}{print}\PY{p}{(}\PY{n}{results}\PY{p}{)}
\end{Verbatim}


    \begin{Verbatim}[commandchars=\\\{\}]
['\#F-ham-mar', 'entrance', 'Enter the Ghost.']
['\#F-ham-hor', 'entrance', 'Enter Ghost againe.']
['\#F-ham-mar', 'exit', 'Exit Ghost.']
['\#F-ham-cla', 'entrance', 'Enter Voltemand and Cornelius.']
['\#F-ham-cla', 'exit', 'Exit Voltemand and Cornelius.']
['\#F-ham-lae', 'entrance', 'Enter Polonius.']
['\#F-ham-hor \#F-ham-mar', 'business', 'within.']
['\#F-ham-pol', 'business', 'The Letter.']
['\#F-ham-pol', 'exit', 'Exit King \& Queen.']
['\#F-ham-ham', 'entrance', 'Enter foure or fiue Players.']
['\#F-ham-ham', 'exit', 'Exit Players.']
['\#F-ham-ham', 'entrance', 'Enter Polonius, Rosincrance, and Guildensterne.']
['\#F-ham-ham', 'exit', 'Exit Polonius.']
['\#F-ham-ger', 'business', 'Sleepes']
['\#F-ham-ham', 'entrance', 'Enter Lucianus.']
['\#F-ham-ham', 'entrance', 'Enter one with a Recorder.']
['\#F-ham-ham', 'MissingType', 'within.']
['\#F-ham-cla', 'entrance', 'Enter Ros. \& Guild.']
['\#F-ham-cla', 'exit', 'Exit Gent.']
['\#F-ham-gmn', 'location', 'within.']

    \end{Verbatim}

    \begin{Verbatim}[commandchars=\\\{\}]

        ---------------------------------------------------------------------------

        ValueError                                Traceback (most recent call last)

        <ipython-input-86-a4442d490155> in <module>()
          6     results.append(checkFor1Result(who, 'MissingWho'))
          7     results.append(checkFor1Result(dirtype, 'MissingType'))
    ----> 8     results.append(" ".join(checkFor1Result(dirtext, 'MissingText').split()))
          9     print(results)
    

        <ipython-input-81-68c835c76142> in checkFor1Result(xpathresult, missing\_value)
          3         howmany = len(xpathresult)
          4         raise ValueError("Your list had " + str(howmany) + " items instead of 1. Shutting down the program,"
    ----> 5                          + "But here's your failed result: " + str(xpathresult))
          6     elif len(xpathresult) == 1:
          7         result = xpathresult[0] \# send the single result back when there's just 1
    

        ValueError: Your list had 7 items instead of 1. Shutting down the program,But here's your failed result: ['Enter ', '\textbackslash{}n                                    ', 'Rosincrane', '\textbackslash{}n                                    ', 'Rosincrance', '\textbackslash{}n                                ', '.']

    \end{Verbatim}

    Oh fun! We can see that we have an error that was hidden from us by our
previous method. This is becuase there's formatted text inside that
element. We can handle the results together before passing it to the
function. But remember that it must be a list!

    \begin{Verbatim}[commandchars=\\\{\}]
{\color{incolor}In [{\color{incolor}87}]:} \PY{k}{for} \PY{n}{stage} \PY{o+ow}{in} \PY{n}{allstages}\PY{p}{:}
             \PY{n}{results} \PY{o}{=} \PY{p}{[}\PY{p}{]}
             \PY{n}{who} \PY{o}{=} \PY{n}{stage}\PY{o}{.}\PY{n}{xpath}\PY{p}{(}\PY{l+s+s1}{\PYZsq{}}\PY{l+s+s1}{../@who}\PY{l+s+s1}{\PYZsq{}}\PY{p}{,} \PY{n}{namespaces} \PY{o}{=} \PY{n}{ns}\PY{p}{)}
             \PY{n}{dirtype} \PY{o}{=} \PY{n}{stage}\PY{o}{.}\PY{n}{xpath}\PY{p}{(}\PY{l+s+s1}{\PYZsq{}}\PY{l+s+s1}{@type}\PY{l+s+s1}{\PYZsq{}}\PY{p}{,} \PY{n}{namespaces} \PY{o}{=} \PY{n}{ns}\PY{p}{)}
             \PY{n}{dirtext} \PY{o}{=} \PY{n}{stage}\PY{o}{.}\PY{n}{xpath}\PY{p}{(}\PY{l+s+s1}{\PYZsq{}}\PY{l+s+s1}{.//text()}\PY{l+s+s1}{\PYZsq{}}\PY{p}{,} \PY{n}{namespaces} \PY{o}{=} \PY{n}{ns}\PY{p}{)}
             \PY{n}{results}\PY{o}{.}\PY{n}{append}\PY{p}{(}\PY{n}{checkFor1Result}\PY{p}{(}\PY{n}{who}\PY{p}{,} \PY{l+s+s1}{\PYZsq{}}\PY{l+s+s1}{MissingWho}\PY{l+s+s1}{\PYZsq{}}\PY{p}{)}\PY{p}{)}
             \PY{n}{results}\PY{o}{.}\PY{n}{append}\PY{p}{(}\PY{n}{checkFor1Result}\PY{p}{(}\PY{n}{dirtype}\PY{p}{,} \PY{l+s+s1}{\PYZsq{}}\PY{l+s+s1}{MissingType}\PY{l+s+s1}{\PYZsq{}}\PY{p}{)}\PY{p}{)}
             \PY{n}{results}\PY{o}{.}\PY{n}{append}\PY{p}{(}\PY{l+s+s2}{\PYZdq{}}\PY{l+s+s2}{ }\PY{l+s+s2}{\PYZdq{}}\PY{o}{.}\PY{n}{join}\PY{p}{(}\PY{n}{checkFor1Result}\PY{p}{(}\PY{p}{[}\PY{l+s+s2}{\PYZdq{}}\PY{l+s+s2}{ }\PY{l+s+s2}{\PYZdq{}}\PY{o}{.}\PY{n}{join}\PY{p}{(}\PY{n}{dirtext}\PY{p}{)}\PY{p}{]}\PY{p}{,} \PY{l+s+s1}{\PYZsq{}}\PY{l+s+s1}{MissingText}\PY{l+s+s1}{\PYZsq{}}\PY{p}{)}\PY{o}{.}\PY{n}{split}\PY{p}{(}\PY{p}{)}\PY{p}{)}\PY{p}{)}
             \PY{n+nb}{print}\PY{p}{(}\PY{n}{results}\PY{p}{)}
\end{Verbatim}


    \begin{Verbatim}[commandchars=\\\{\}]
['\#F-ham-mar', 'entrance', 'Enter the Ghost.']
['\#F-ham-hor', 'entrance', 'Enter Ghost againe.']
['\#F-ham-mar', 'exit', 'Exit Ghost.']
['\#F-ham-cla', 'entrance', 'Enter Voltemand and Cornelius.']
['\#F-ham-cla', 'exit', 'Exit Voltemand and Cornelius.']
['\#F-ham-lae', 'entrance', 'Enter Polonius.']
['\#F-ham-hor \#F-ham-mar', 'business', 'within.']
['\#F-ham-pol', 'business', 'The Letter.']
['\#F-ham-pol', 'exit', 'Exit King \& Queen.']
['\#F-ham-ham', 'entrance', 'Enter foure or fiue Players.']
['\#F-ham-ham', 'exit', 'Exit Players.']
['\#F-ham-ham', 'entrance', 'Enter Polonius, Rosincrance, and Guildensterne.']
['\#F-ham-ham', 'exit', 'Exit Polonius.']
['\#F-ham-ger', 'business', 'Sleepes']
['\#F-ham-ham', 'entrance', 'Enter Lucianus.']
['\#F-ham-ham', 'entrance', 'Enter one with a Recorder.']
['\#F-ham-ham', 'MissingType', 'within.']
['\#F-ham-cla', 'entrance', 'Enter Ros. \& Guild.']
['\#F-ham-cla', 'exit', 'Exit Gent.']
['\#F-ham-gmn', 'location', 'within.']
['\#F-ham-cla', 'entrance', 'Enter Rosincrane Rosincrance .']
['\#F-ham-cla', 'business', 'A noise within.']
['\#F-ham-sai', 'business', 'Reads the Letter.']
['\#F-ham-cla', 'entrance', 'Enter a Messenger.']
['\#F-ham-cla', 'exit', 'Exit Messenger']
['\#F-ham-clo.1', 'business', 'Sings.']
['\#F-ham-clo.1', 'business', 'sings.']
['\#F-ham-clo.1', 'business', 'sings.']
['\#F-ham-ham', 'entrance', 'Enter King, Queen, Laertes, and a Coffin, with Lords attendant.']
['\#F-ham-lae', 'business', 'Leaps in the graue.']
['\#F-ham-ham', 'business', 'March afarre off, and shout within.']

    \end{Verbatim}

    This in coming from this item on line 5934:

\begin{Shaded}
\begin{Highlighting}[]
\KeywordTok{<stage}\OtherTok{ rend=}\StringTok{"italic center"}\OtherTok{ type=}\StringTok{"entrance"}\KeywordTok{>}\NormalTok{Enter }\KeywordTok{<choice>}
                                    \KeywordTok{<orig>}\NormalTok{Rosincrane}\KeywordTok{</orig>}
                                    \KeywordTok{<corr>}\NormalTok{Rosincrance}\KeywordTok{</corr>}
                                \KeywordTok{</choice>}\NormalTok{.}\KeywordTok{</stage>}
\end{Highlighting}
\end{Shaded}

    If we were really wanting to do something with this, we would want to
deal with that. But we are not, so we can let it go.

    \section{Exporting results}\label{exporting-results}

Now that we have some nice tabular results, we can export this out to a
file!

We can use the CSV module to write out a nice CSV. This will
automatically sanitize our results for us. You'll need to have two
things:

\begin{enumerate}
\def\labelenumi{\arabic{enumi}.}
\tightlist
\item
  a list with the string values of the headers that you want (one string
  per header).

  \begin{itemize}
  \tightlist
  \item
    \texttt{{[}\textquotesingle{}speakerID\textquotesingle{},\ \textquotesingle{}directionType\textquotesingle{},\ \textquotesingle{}directionText\textquotesingle{}{]}}
  \end{itemize}
\item
  a list of lists, where each list contains a single row of data. Each
  element will become a different column.

  \begin{itemize}
  \tightlist
  \item
    we are already making our row lists, we just need to collect them.
  \end{itemize}
\end{enumerate}

    \begin{Verbatim}[commandchars=\\\{\}]
{\color{incolor}In [{\color{incolor}88}]:} \PY{n}{allresults} \PY{o}{=} \PY{p}{[}\PY{p}{]}
         
         \PY{k}{for} \PY{n}{stage} \PY{o+ow}{in} \PY{n}{allstages}\PY{p}{:}
             \PY{n}{results} \PY{o}{=} \PY{p}{[}\PY{p}{]}
             \PY{n}{who} \PY{o}{=} \PY{n}{stage}\PY{o}{.}\PY{n}{xpath}\PY{p}{(}\PY{l+s+s1}{\PYZsq{}}\PY{l+s+s1}{../@who}\PY{l+s+s1}{\PYZsq{}}\PY{p}{,} \PY{n}{namespaces} \PY{o}{=} \PY{n}{ns}\PY{p}{)}
             \PY{n}{dirtype} \PY{o}{=} \PY{n}{stage}\PY{o}{.}\PY{n}{xpath}\PY{p}{(}\PY{l+s+s1}{\PYZsq{}}\PY{l+s+s1}{@type}\PY{l+s+s1}{\PYZsq{}}\PY{p}{,} \PY{n}{namespaces} \PY{o}{=} \PY{n}{ns}\PY{p}{)}
             \PY{n}{dirtext} \PY{o}{=} \PY{n}{stage}\PY{o}{.}\PY{n}{xpath}\PY{p}{(}\PY{l+s+s1}{\PYZsq{}}\PY{l+s+s1}{.//text()}\PY{l+s+s1}{\PYZsq{}}\PY{p}{,} \PY{n}{namespaces} \PY{o}{=} \PY{n}{ns}\PY{p}{)}
             \PY{n}{results}\PY{o}{.}\PY{n}{append}\PY{p}{(}\PY{n}{checkFor1Result}\PY{p}{(}\PY{n}{who}\PY{p}{,} \PY{l+s+s1}{\PYZsq{}}\PY{l+s+s1}{MissingWho}\PY{l+s+s1}{\PYZsq{}}\PY{p}{)}\PY{p}{)}
             \PY{n}{results}\PY{o}{.}\PY{n}{append}\PY{p}{(}\PY{n}{checkFor1Result}\PY{p}{(}\PY{n}{dirtype}\PY{p}{,} \PY{l+s+s1}{\PYZsq{}}\PY{l+s+s1}{MissingType}\PY{l+s+s1}{\PYZsq{}}\PY{p}{)}\PY{p}{)}
             \PY{n}{results}\PY{o}{.}\PY{n}{append}\PY{p}{(}\PY{l+s+s2}{\PYZdq{}}\PY{l+s+s2}{ }\PY{l+s+s2}{\PYZdq{}}\PY{o}{.}\PY{n}{join}\PY{p}{(}\PY{n}{checkFor1Result}\PY{p}{(}\PY{p}{[}\PY{l+s+s2}{\PYZdq{}}\PY{l+s+s2}{ }\PY{l+s+s2}{\PYZdq{}}\PY{o}{.}\PY{n}{join}\PY{p}{(}\PY{n}{dirtext}\PY{p}{)}\PY{p}{]}\PY{p}{,} \PY{l+s+s1}{\PYZsq{}}\PY{l+s+s1}{MissingText}\PY{l+s+s1}{\PYZsq{}}\PY{p}{)}\PY{o}{.}\PY{n}{split}\PY{p}{(}\PY{p}{)}\PY{p}{)}\PY{p}{)}
             \PY{n}{allresults}\PY{o}{.}\PY{n}{append}\PY{p}{(}\PY{n}{results}\PY{p}{)}
\end{Verbatim}


    \begin{Verbatim}[commandchars=\\\{\}]
{\color{incolor}In [{\color{incolor}89}]:} \PY{n}{headers} \PY{o}{=} \PY{p}{[}\PY{l+s+s1}{\PYZsq{}}\PY{l+s+s1}{speakerID}\PY{l+s+s1}{\PYZsq{}}\PY{p}{,} \PY{l+s+s1}{\PYZsq{}}\PY{l+s+s1}{directionType}\PY{l+s+s1}{\PYZsq{}}\PY{p}{,} \PY{l+s+s1}{\PYZsq{}}\PY{l+s+s1}{directionText}\PY{l+s+s1}{\PYZsq{}}\PY{p}{]}
\end{Verbatim}


    \begin{Verbatim}[commandchars=\\\{\}]
{\color{incolor}In [{\color{incolor}90}]:} \PY{k+kn}{import} \PY{n+nn}{csv}
         
         \PY{n}{outfile} \PY{o}{=} \PY{n+nb}{open}\PY{p}{(}\PY{l+s+s1}{\PYZsq{}}\PY{l+s+s1}{stagedirections.csv}\PY{l+s+s1}{\PYZsq{}}\PY{p}{,} \PY{l+s+s1}{\PYZsq{}}\PY{l+s+s1}{w}\PY{l+s+s1}{\PYZsq{}}\PY{p}{)}
         \PY{n}{csvout} \PY{o}{=} \PY{n}{csv}\PY{o}{.}\PY{n}{writer}\PY{p}{(}\PY{n}{outfile}\PY{p}{)}
         \PY{n}{csvout}\PY{o}{.}\PY{n}{writerow}\PY{p}{(}\PY{n}{headers}\PY{p}{)}
         \PY{n}{csvout}\PY{o}{.}\PY{n}{writerows}\PY{p}{(}\PY{n}{allresults}\PY{p}{)}
\end{Verbatim}


    And we're done!


    % Add a bibliography block to the postdoc
    
    
    
    \end{document}
